\documentclass[10pt,a4paper]{article}
\usepackage[lmargin=3.0cm,rmargin=3.0cm,tmargin=3.0cm,bmargin=3.0cm,head=2cm,headsep=0.5cm]{geometry}
\setlength\parindent{0pt}
\usepackage[utf8]{inputenc}
\usepackage{amssymb}
\usepackage{color}
\usepackage{colortbl}
\usepackage[ngerman]{babel}
\usepackage{here} 
\usepackage{graphicx}
\usepackage{wrapfig}
\usepackage{floatflt}
\usepackage[singlespacing]{setspace}
\usepackage{float}
\graphicspath{{./images/} }
\usepackage[hang]{footmisc}
\linespread{1.5}
\usepackage{listings} \lstset{numbers=left, numberstyle=\tiny, numbersep=5pt} 
\definecolor{code}{rgb}{0.94, 0.97, 1.0}
\definecolor{red}{rgb}{1.0, 0.03, 0.0}
\definecolor{green}{rgb}{0.0, 0.65, 0.31}
\usepackage{hyperref}
\usepackage{caption}
\usepackage[numbers]{natbib} % Wird für Zitierstil dinat benötigt; square ist für Zitation in eckigen Klammern und numbers für Zitation mit Nummern
\lstset{
	  basicstyle=\ttfamily,
	  breaklines=true,
	  numberstyle=\footnotesize,
      numbersep=5pt,   
	  literate={Ö}{{\"O}}1 {Ä}{{\"A}}1 {Ü}{{\"U}}1 {ß}{{\ss}}2 {ü}{{\"u}}1 {ä}{{\"a}}1 {ö}{{\"o}}1 {µ}{\textmu}1,
      columns=fullflexible,
      showstringspaces=false,
      commentstyle={\color{green}},
      keywordstyle=\color{blue},
      stringstyle=\color{red},
    %  numbers=none,
      extendedchars=\true,
      tabsize=4,
      breaklines=true,
      breakatwhitespace=true,
      language=Java,
      backgroundcolor=\color{code}} 

\title{IT-Projekt}

\begin{document}
\begin{center}
\huge
Arbeitsverteilung
\end{center}
\begin{singlespace}
\tableofcontents
\end{singlespace}
\section{Entwicklung Server}
Die Serveranwendung wurde komplett von \textbf{Baris Akdag} entwickelt.
\section{Entwicklung App}
\begin{tabular}{ | l | p{5,5cm} |}
	\hline
	\textbf{Klasse} & \textbf{Entwickler}\\    
	\hline
	Accelerometer & Christopher Althaus\\    
    \hline
	AccelSample & Christopher Althaus\\
	\hline	
	Localizer & Niklas Schäfer\\
	\hline	
	BootCompletedReceiver & Jürgen Hetzel\\
	\hline	
	ConnectionChangeReceiver & Niklas Schäfer\\
	\hline	
	GcmBroadcastReceiver & Christopher Althaus\\
	\hline
	DeviceMap & Niklas Schäfer\\
	\hline
	Info & Christopher Althaus\newline Niklas Schäfer \\
	\hline
	Main & Christopher Althaus\newline Niklas Schäfer\\
	\hline
	Settings & Niklas Schäfer\\
	\hline
	BackgroundService & Christopher Althaus\newline Niklas Schäfer\\
	\hline
	GcmIntentService & Christopher Althaus\\
	\hline
	NotificationsService & Niklas Schäfer\\
	\hline
	AndroidManifest.xml und verschiedene Layout xml Files & Christopher Althaus\newline Niklas Schäfer\newline Jürgen Hetzel\\
	\hline
\end{tabular}
\section{Taskzusammenfassung}
\begin{tabular}{ | l | p{11cm} |}
	\hline
	\textbf{Bearbeiter} & \textbf{Aufgaben}\\    
	\hline
	Baris Akdag & 	\textbf{komplette Serveranwendung}
					\newline Testing (Serveranwendung, Erdbebenerkennung)		\\
	\hline	
	Christopher Althaus & 	- Erstellen der Basis-App und grobe Strukturierung\newline
							- Auswertung der Beschleunigungssensordaten (Erdbebenerkennung)\newline 
							- Visualisierung der Beschleunigungssensordaten (Diagramm)\newline
							- Google Cloud Messaging (inkl. Erdbebenalarm Notifications)\newline 
							- 	Backgroundprozess (Entwicklung)\newline 
							- 	User Interface (Tabbed Layout, Beschleunigungsensoren Graph)\newline 
							- 	Testing (Grundlegende App Funktionen, Erdbebenerkennung, Lokalisierung, Akkuverbrauch)\\    
    \hline
	Niklas Schäfer & 		- 	Lokalisierung (komplette Entwicklung)\newline 
							- 	Device Map (komplette Entwicklung)\newline 
							- 	User Interface (Oberflächenelemente, Device Map, Settings)\newline 
							- 	Backgroundprozess (Erweiterung um Intervallsteuerung und http-requests bzgl. Lokalisierung)\newline 
							- 	Notifications (bzgl. Location Provider und Internetverbindung)\newline 
							- 	App Settings\newline 
							- 	Reaktion auf verschiedene Ereignisse (Location Provider Deaktivierung/Aktivierung, keine Internetverbindung, usw.)\newline 
							- 	Kommunikation GFZ Potsdam und BGR\newline 
							- 	Testing(Grundlegende App Funktionen, Lokalisierung, Akkuverbrauch, Erdbebenerkennung) \\
	\hline	
	Jürgen Hetzel & 			- 	Layout für Tablets\newline 
							- 	Sicherstellung der Backgroundprozessaktivität\newline 
							- 	Refactoring\newline 
							- 	Testing (Erdbebenerkennung, User Interface)\newline 
							- 	Recherchearbeiten\newline
							-	Mitwirkung an der Funktion \glqq Einstellungen zurücksetzen\grqq\\
	\hline	
	Benjamin Brandt & 		- Testing (gesamtes System (sowohl Serveranwendung, als auch App)\newline 
							- Recherchearbeiten)	\\
	\hline	
\end{tabular}
\end{document}
