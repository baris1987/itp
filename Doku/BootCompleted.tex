\section[Starten des Services für die Erdbebenerkennung (Jürgen Hetzel)]{Das Starten des Services für die Erdbebenerkennung \begin{tiny} (Jürgen
Hetzel)\end{tiny}}

Der Hintergrundservice ist für die Funktion der App essentiell. Infolgedessen muss dieser dauerhaft in Betrieb sein. Um das zu gewährleisten, wird ein \emph{Broadcast Receiver} eingesetzt, der den Hintergrundservice nach erfolgreichem Boot-Vorgang mittels Intent startet. Der Hintergrundservice kann zwar durch Eliminieren des Prozesses beendet werden, wird jedoch bei Neustart des Smartphones wieder aktiviert.

\begin{lstlisting}[caption={Der BootCompleted-Receiver der App},label=lst:BootCompletedReceiver]
public class BootCompletedReceiver extends BroadcastReceiver {
	@Override
	public void onReceive(Context context, Intent intent) {		
		Intent backgroundServiceIntent = new Intent(context, BackgroundService.class);
		context.startService(backgroundServiceIntent);
		Log.i("QuakeDetect", "Run background service");
	}
}
\end{lstlisting}


Mit Hilfe von \emph{android:enabled = "true"} wird die Instanzierung des Receivers ermöglicht. Die Eigenschaft \emph{android:exported = "false"} verhindert, dass der Receiver Nachrichten von anderen Applikationen erhält. Des Weiteren wird ein Filter eingesetzt, so dass der Receiver nur auf die Systemnachricht für erfolgreichen Boot-Vorgang reagiert.

\begin{lstlisting}[caption={Anpassungen im Manifest hinsichtlich des BootCompletedReceivers},label=lst:BootCompletedManifest]
<receiver android:name="com.th.nuernberg.quakedetec.receivers.BootCompletedReceiver"
    android:enabled="true" 
     android:exported="false">
    <intent-filter>
        	<action android:name="android.intent.action.BOOT_COMPLETED" />
    </intent-filter>
</receiver>
\end{lstlisting}