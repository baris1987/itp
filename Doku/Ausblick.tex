\section{Ausblick}
Die aktuelle Implementierung ist noch in einem Alpha Status.
Da uns weitreichendes Know-how im Bereich Erdbebenerkennung fehlte und ebenfalls technische Hilfsmittel, wie beispielsweise Rüttelplatten, nicht zur Verfügung standen, ist die Erkennung derzeitig nicht ausreichend präzise genug, um Erdbeben zuverlässig zu erkennen und Fehlmeldungen auszuschließen. 
Daher haben wir uns an die Bundesanstalt für Geowissenschaften und Rohstoffe gewandt, welche uns an das Geoforschungszentrum Potsdam verwies.
Dort hatten wir Kontakt zu Herrn Stefano Parolai (Head of the Centre for Early Warning) hergestellt, der einige unserer Fragen beantwortete.
Weiterhin arbeitet einer der Partner des GFZ Potsdam an einem Projekt, das in eine ähnliche Richtung führt wie dieses. Daher ist das GFZ Potsdam daran interessiert ein Skype Meeting abzuhalten. 
In diesem Meeting können weitere Fragen und eventuell gemeinsame Vorgehensweisen besprochen werden.
Sollte es uns möglich sein die App zuverlässiger und präziser in der Erkennung zu implementieren, kann durchaus über eine Veröffentlichung nachgedacht werden.   