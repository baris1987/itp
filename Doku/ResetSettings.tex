\section[Zurücksetzen der Einstellungen (Jürgen Hetzel)]{Zurücksetzen der Einstellungen \begin{tiny} (Jürgen
Hetzel)\end{tiny}}

Für den Fall, dass ein Benutzer ungewollt die Einstellungen veränderte, sollte die Möglichkeit vorhanden sein, die Einstellungen auf die Standard-Werte zurückzusetzen.
Um dies zu realisieren, wird zunächst mittels Key \emph{application\textunderscore restore} auf die entsprechende Einstellung zurückgegriffen. Im Anschluß daran, wird für diese ein \emph{OnClickListener} angelegt. Für die Benutzer-Interaktion ist ein Dialog vorgesehen. Zu diesem Zweck, gibt es eine adäquate Android-Klasse, inklusive \emph{Builder}. Die benötigten \emph{Strings} werden über die zugehörige XML-Datei eingebunden.
\begin{lstlisting}[caption={Listener und Alert Dialog für das Zurücksetzen der Einstellungen 1},label=lst:resetSettingsListener1]
Preference restorePref = (Preference) findPreference("application_restore");

restorePref.setOnPreferenceClickListener(new OnPreferenceClickListener() {
    public boolean onPreferenceClick(Preference preference) {
   	 AlertDialog.Builder alertDialogBuilder = new AlertDialog.Builder(Settings.getSettingsObject());
		
	 alertDialogBuilder.setTitle(getResources().getString(R.string
	    .settings_reset));			
	 alertDialogBuilder
	 	.setMessage(getResources().getString(R.string.settings_reset_sure))
		.setCancelable(false)
\end{lstlisting}
Der Dialog enthält die Buttons \emph{OK}	 und \emph{Cancel}. Für diese wird jeweils ein \emph{onClickListener} benötigt. Innerhalb des OK-Button-Listeners wird mit Hilfe von \emph{getSettingsObject} auf die derzeitige Instanz von Settings zugegriffen und anschließend die Methode \emph{resetPrefsToDefault} aufgerufen. Diese wird in Listing \ref{lst:resetSettingsMethod} erläutert. Der Cancel-Button ist hier nicht aufgeführt, da dessen Betätigung keine zusätzliche Aktion bewirkt. Erst mit der Create-Methode der Builder-Klasse, wird tatsächlich ein Alert-Dialog mit der festgelegten Konfiguration erstellt und schließlich mit \emph{show()} angezeigt.
 \begin{lstlisting}[caption={Listener und Alert Dialog für das Zurücksetzen der Einstellungen 2},label=lst:resetSettingsListener2]		
		.setPositiveButton(getResources().getString(R.string.btn_ok),
		   new DialogInterface.OnClickListener() {
			public void onClick(DialogInterface dialog,int id) 
			{
				isRestore = true;
				Settings.getSettingsObject().resetPrefsToDefault();
			}
		});
		
	 AlertDialog alertDialog = alertDialogBuilder.create();
	 alertDialog.show();
   	 return true;
    }
});
\end{lstlisting}

Nachfolgend wird anfangs mittels \emph{PreferenceManager} auf die Einstellungen zugegriffen und für diese ein Editor erzeugt. Der Editor dient zur Löschung der Werte, welche jedoch erst nach \emph{commit} übertragen werden. Im Anschluß daran, werden die festgelegten Werte aus dem definierten XML-File verwendet und schließlich die App neu gestartet.
\begin{lstlisting}[caption={Die Methode für das Zurücksetzen der Einstellungen},label=lst:resetSettingsMethod]
public void resetPrefsToDefault()
{
	PreferenceManager.getDefaultSharedPreferences(this).edit().clear().commit();	
	PreferenceManager.setDefaultValues(this, R.xml.preferences, true);
	restartApp();
}
\end{lstlisting}

