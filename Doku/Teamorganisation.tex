\section{Teamorganisation}
Die Projektgruppe besteht aus Niklas Schäfer, Baris Akdag, Christopher Althaus, Benjamin Brandt sowie Jürgen Hetzel. Innerhalb der Gruppe sind zu Beginn die verschiedenen Aufgabengebiete nach Interessen und Fähigkeiten des einzelnen verteilt worden.\\
Baris Akdag verfügte im Vorfeld über Fachkenntnisse in den Bereichen WebServices und Datenbanken. Er übernahm die Entwicklung des gesamten WebServices inklusive Datenbank und Bereitstellung.\\
Durch die Erfahrung von Christopher Althaus im Bereich Android Programmierung bot er sich neben Niklas Schäfer an, die Android Applikation zu entwicklen.\\
Dabei übernahm Niklas Schäfer als Hauptaufgaben die Lokalisierung der Geräte inklusive der Sicherstellung aktivierter Standortbestimmung auf den Smartphones, sowie die Einbindung und Weiterentwicklung der Google Maps Karte. Weiterhin implementierte er die Einstellungen (Settings View und deren Funktion) und arbeitete am User Interface der App.\\ 
Christopher Althaus widmete sich neben der groben Strukturierung der App, hauptsächlich um die Aufgabengebiete rund um den Beschleunigungssensor und um die Benutzerbenachrichtigung im Falle eines Erdbebens. Diese Aufgaben umfassten zum Einen die Erdbebenerkennung innerhalb der Applikation und zum anderen die Einbindung eines Diagramms zur Visualisierung der Beschleunigungsdaten.\\
Jürgen Hetzel und Benjamin Brand übernahmen während des Projektablaufs einen Großteil der Literaturrecherche. Ebenso kümmerte sich Jürgen Hetzel zum Ende des Projekts um das Refactoring der Android Applikation.
Da Benjamin Brand über eine große Auswahl von Geräten verfügte, übernahm er zusätzlich das Testen der Anwendung.\\
Über den gesamten Zeitraum der Bearbeitung ist eine enge Zusammenarbeit und gute Kommunikation Grundlage für ein erfolgreiches Umsetzen des Projekts gewesen.
\newpage